\section{Introduction}
\MBSim{} is a simulation tool to analyse the dynamic phenomenons of dynamical systems. Its root is the modelling of nonsmooth multibody system explaining the program name \MBSim{}. The mathematical background has been developed over years at the Institute of Applied Mechanics of the Technische Universit\"at M\"unchen. The last summary concerning rigid body dynamics was given in the PhD thesis and the lecture of Martin F\"org~\cite{Foer09,Foer07}. The PhD thesis of Roland Zander~\cite{Zan09} introduces the theory of flexible bodies. Ref.~\cite{Zan08} shows an overview about the research at the institute in the last decades concerning nonsmooth mechanics. This reference also includes simulation results of academic and industrial examples. Extensions regarding hydraulics and signal processing as well as parallelisation and cosimulation are unique in the field of nonsmooth dynamical systems.\par
The goal of this introduction is to motivate the use of \MBSim{}. It shows the installation of the necessary program parts, describes basically its components and program flow and gives some examples. You find it in the Download section of the \MBSim{} webpage, where you can also find binary releases for Linux and Windows as well as~\cite{Zan08}. For more information, e.g. Doxygen description and current build status, visit the {\href{http://www4.amm.mw.tu-muenchen.de:8080/mbsim-env/}{\textsf{Central Build System}}}.

