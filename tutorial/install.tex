\section{Installation}
This section summarizes the necessary steps to install \MBSim{} \cite{Foer06}.

%%------------------------------------------------------------ SUBSECTION ---------------------
\subsection{Where To Find the Source Code}
The source code of \MBSim{} together with some examples, the necessary \FMatVec{} library, a \HDF{} wrapper for output and the visualisation program \OpenMBV{} can be found at \url{www.berlios.de} using subversion code administration\footnote{SVN Quick Reference Card: \url{http://www.cs.put.poznan.pl/csobaniec/edu/svn-refcard.pdf}}. Further, one needs \HDF{} from \url{http://www.hdfgroup.org}. Everything is placed under \href{http://www.gnu.org/licenses/lgpl.html}{LGPL}\footnote{see file~\texttt{COPYING} in the root directory of the specific source code}.\par
In the following it is assumed, that a directory~\texttt{MBSim} and a directory \texttt{MBSim/Install} has been created in the \texttt{\$HOME} path of the Linux operating system. 

%%------------------------------------------------------------ SUBSECTION ---------------------
\subsection{Installation Procedures}
All projects depend on PKG package administration. That is why the file \texttt{\$HOME/.bashrc} has to be extended with
\begin{verbatim}
 export PKG_CONFIG_PATH=
        "$HOME/MBSim/Install/lib/pkgconfig/:$PKG_CONFIG_PATH"
 export LD_LIBRARY_PATH=
        "$HOME/MBSim/Install/lib/:$LD_LIBRARY_PATH"
\end{verbatim}
For the installation of the specific projects always the same \emph{procedures} have to be applied. They are summarised in the following.

\subsubsection{Installation}
\begin{itemize}
	\item \textsc{automake}:
	\begin{itemize}
		\item[] \begin{verbatim}aclocal\end{verbatim}
		\item[] \texttt{autoheader}
		\item[] \texttt{autoconf}
		\item[] \begin{verbatim}libtoolize -c --force\end{verbatim}
		\item[] \begin{verbatim}automake -a -c --force\end{verbatim}
	\end{itemize}
	\item \textsc{configure}: 
	\begin{itemize}
		\item[] \begin{verbatim}./configure --prefix=$HOME/MBSim/Install F77=gfortran \end{verbatim}
        \item[] possibly FLAGS for debug information
		        \begin{verbatim} CFLAGS="-g3 -O0" CXXFLAGS="-g3 -O0" F77FLAGS="-g3 -O0" FFLAGS="-g3 -O0" \end{verbatim}
		\item[] possibly project depending FLAGS
	\end{itemize}
	\item \textsc{install}
	\begin{itemize}	
		\item \begin{verbatim}make\end{verbatim}
		\item \begin{verbatim}make install\end{verbatim}
	\end{itemize}
\end{itemize}
All procedures belong to the GNU-Build-System (cf.~Sec.~\ref{sec:gnu}).\par
It is reasonable to write an executable script file invoking the procedures.

\subsubsection{Reinstallation}
The procedure \textsc{reinstall}
\begin{verbatim}
 ./config.status -recheck
 ./config.status
 make clean
 make install
\end{verbatim}
newly installs a project.\par
For restoring a not-configured version of the project
\begin{verbatim}
 make maintainer-clean
\end{verbatim}
is used. After that \textsc{configure} has to be invoked again.

\subsubsection{Uninstallation}
For uninstall
\begin{verbatim}
 make clean
 make uninstall
\end{verbatim}
has to be called in all directories.

%%------------------------------------------------------------ SUBSECTION ---------------------
\subsection{\FMatVec{}}
\FMatVec{} is a library for fast matrix-vector evaluations based on LAPack with the possibility to use ATLAS, which is faster for large system dimensions because of internal parallelisation.\\
For the installation the following instructions have to be completed:
\begin{verbatim}
 cd $HOME/MBSim
 svn checkout http://svn.berlios.de/svnroot/repos/fmatvec/trunk fmatvec
 cd $HOME/MBSim/fmatvec
\end{verbatim}
Continue with the procedure \textsc{automake}.\par
Then, the procedure \textsc{configure} is used selecting the FLAGS
\begin{verbatim}
 --with-blas-lib-prefix=PFX (prefix, where the BLAS lib is
     installed, when ATLAS is not used)
 --with-lapack-lib-prefix=PFX (prefix, where the LAPACK lib is
     installed, when ATLAS is not used)
\end{verbatim}
if LAPack is not located in a standard search path and selecting
\begin{verbatim}
 --enable-atlas (use ATLAS)
 --with-atlas-inc-prefix=PFX (prefix, where the ATLAS includes 
     are installed, when ATLAS is used)
 --with-atlas-lib-prefix=PFX (prefix, where the ATLAS libs are
     installed, when ATLAS is used)
\end{verbatim}
if ATLAS should be used instead of LAPack. The FLAGS
\begin{verbatim}
 --with-allocator-class=__gnu_cxx::new_allocator
 --with-allocator-header=ext/new_allocator.h
\end{verbatim}
are necessary to use a standard thread-safe memory allocator instead of the \FMatVec{} built-in allocator.\par
The code can be compiled and installed with a Doxygen HTML class documentation by \texttt{make doc} and the procedure \textsc{install}.

%%------------------------------------------------------------ SUBSECTION ---------------------
\subsection{\HDF}
\HDF{} is a hierarchical data format enabling the effective administration of plot and visualisation data. It can be downloaded as source code from \url{http://www.hdfgroup.org/HDF5/} with at least version 1.8.2.\par
Extract the source archive to \texttt{\$HOME/MBSim/hdf5}.\par
Change to \texttt{\$HOME/MBSim/hdf5}.\par
Use the procedure \textsc{configure} with the additional FLAG
\begin{verbatim}
 --enable-cxx
\end{verbatim}
Compilation is done with the procedure \textsc{install}.\par
A \HDF{} wrapper makes it possible to use \HDF{} very easily. It is available by
\begin{verbatim}
 cd $HOME/MBSim
 svn checkout http://svn.berlios.de/svnroot/repos/hdf5serie/trunk HDF5Serie
\end{verbatim}
For having \MBSim{} creating \HDF{} files invoke
\begin{verbatim}
 cd $HOME/MBSim/HDF5Serie/hdf5serie
\end{verbatim}
as well as the procedures \textsc{automake, configure}, \texttt{make doc} and \textsc{install} for installation and creation of a Doxygen HTML class documentation.\par
For convenient plotting of \HDF{} files it is assumed that Qwt with version 5 or newer is installed (cf. Sec.~\ref{sec:third_party}).\par
Invoke 
\begin{verbatim}
 ssh diesel (only at the institute)
 export PKG_CONFIG_PATH=/home/OpenMBV/local/lib/pkgconfig:
    $HOME/MBSim/Install/lib/pkgconfig (only at the institute)
 cd $HOME/MBSim/HDF5Serie/h5plotserie
\end{verbatim}
as well as the procedures \textsc{automake, configure}, \texttt{make doc} and \textsc{install} for installation and creation of a Doxygen HTML class documentation. Then,
\begin{verbatim}
 exit (only at the institute)
\end{verbatim}
Last, \texttt{.bashrc} can be extended with
\begin{verbatim}
alias h5lsserie="$HOME/MBSim/Install/bin/h5lsserie"
alias h5dumpserie="$HOME/MBSim/Install/bin/h5dumpserie"
alias h5plotserie="$HOME/MBSim/Install/bin/h5plotserie"
\end{verbatim}
to gain overall access to the commands \texttt{h5lsserie}, \texttt{h5dumpserie} and \texttt{h5plotserie}.

%%------------------------------------------------------------ SUBSECTION ---------------------
\subsection{\OpenMBV{}}
\OpenMBV{} visualises \MBSim{} simulations using XML and \HDF{} in a coinciding hierarchical structure. The installation consists of three steps: first the XML utils have to be installed, then \OpenMBV{} has to be build for visualisation, and third \MBSim{} needs \textsf{OpenMBV-C++Interface} to create standard data for \OpenMBV{} using C++ programs. The source code is available by
\begin{verbatim}
 cd $HOME/MBSim
 svn checkout http://svn.berlios.de/svnroot/repos/openmbv/trunk OpenMBV
\end{verbatim}

\subsubsection{XML Utils}
It is assumed that Octave with version 3.0 or newer is installed (cf. Sec.~\ref{sec:third_party}).\par
Then,
\begin{verbatim}
 cd $HOME/MBSim/OpenMBV/mbxmlutils
\end{verbatim} 
and use the procedures \textsc{automake}, \textsc{configure} with FLAG \texttt{\-\-with-octave-path} and \textsc{install} for installation of a XML interface. 

\subsubsection{\OpenMBV{}}
There is a static Linux binary available at \url{www.berlios.de} being updated from time to time.\par
For the installation of a static visualisation using always the newest source files it is assumed that
\begin{itemize}
\item Coin3d with version 3 or newer 
\item hdf5 with version 1.8.2 or newer 
\item Qt with version 4.4 or newer 
\item HDF5Serie 
\item SoQt with version 1.4.1 or newer 
\item Qwt with version 5 or newer 
\end{itemize}
is installed (cf. Sec.~\ref{sec:third_party}).\par
With
\begin{verbatim}
 ssh diesel (only at the institute)
 export PKG_CONFIG_PATH=/home/OpenMBV/local/lib/pkgconfig:
    $HOME/MBSim/Install/lib/pkgconfig (only at the institute)
 cd $HOME/MBSim/OpenMBV/openmbv
\end{verbatim} 
and the procedures \textsc{automake, configure}, \texttt{make doc} and \textsc{install} a static build of the viewer together with an Doxygen HTML class documentation completes the installation. Then,
\begin{verbatim}
 exit (only at the institute)
\end{verbatim}
Last, \texttt{.bashrc} can be extended with
\begin{verbatim}
 alias openmbv="$HOME/MBSim/Install/bin/openmbv"
\end{verbatim}
to gain overall access to the command \texttt{openmbv}, which should be used only locally because of network protocols not providing the necessary X.org requirements.

\subsubsection{OpenMBV-C++Interface}
It is assumed that 
\begin{itemize}
\item hdf5 with version 1.8.2 or newer
\item HDF5Serie 
\end{itemize}
is installed.\par
Invoke
\begin{verbatim}
 cd $HOME/MBSim/OpenMBV/openmbvcppinterface
\end{verbatim} 
and the procedures \textsc{automake, configure}, \texttt{make doc} and \textsc{install} for installation and creation of a Doxygen HTML class documentation. 

%%------------------------------------------------------------ SUBSECTION ---------------------
\subsection{\MBSim}
Necessary for the installation of \MBSim{} are
\begin{itemize}
\item \FMatVec{}
\item \OpenMBV{}-C++-Interface
\end{itemize}
For installation of \MBSim{} one types
\begin{verbatim}
 cd $HOME/MBSim
 svn checkout http://svn.berlios.de/svnroot/repos/mbsim/trunk mbsim
 cd $HOME/MBSim/mbsim/kernel
\end{verbatim}
Invoke the procedures \textsc{automake, configure}, \texttt{make doc} and \textsc{install} to install the basic module and to create a Doxygen HTML class documentation. 
\begin{verbatim}
 cd $HOME/MBSim/mbsim/modules/mbsimControl
\end{verbatim}
Invoke the procedures \textsc{automake, configure}, \texttt{make doc} and \textsc{install} to install the signal processing and control module and to create a Doxygen HTML class documentation. 
\begin{verbatim}
 cd $HOME/MBSim/mbsim/modules/mbsimHydraulics
\end{verbatim}
Invoke the procedures \textsc{automake, configure}, \texttt{make doc} and \textsc{install} to install the hydraulics module and to create a Doxygen HTML class documentation. 
\begin{verbatim}
 cd $HOME/MBSim/mbsim/modules/mbsimElectronics
\end{verbatim}
Invoke the procedures \textsc{automake, configure}, \texttt{make doc} and \textsc{install} to install the electronics module and to create a Doxygen HTML class documentation. 
\begin{verbatim}
 cd $HOME/MBSim/mbsim/modules/mbsimPowerTrain
\end{verbatim}
Invoke the procedures \textsc{automake, configure}, \texttt{make doc} and \textsc{install} to install the module for gears and to create a Doxygen HTML class documentation. 
\begin{verbatim}
 cd $HOME/MBSim/mbsim/mbsimxml
\end{verbatim}
Invoke the procedures \textsc{automake, configure} and \textsc{install} to install the xml module.
\begin{verbatim}
 cd $HOME/MBSim/mbsim/kernel
\end{verbatim}
Invoke \texttt{make xmldoc} to create an HTML class documentation of the XML interface. 

%%------------------------------------------------------------ SUBSECTION ---------------------
\subsection{MBSimAddOns}
At the institute further modules can be found in \texttt{MBSimAddOn}: mechanical components~(\texttt{mbsimMechAdd}), hydraulic components~(\texttt{mbsimHydraulics}), and an interface for co-simulationen~(\texttt{mbsimCosim}). For installation one types
\begin{verbatim}
 svn checkout http://zeiss/repos/SVN--rep//AM-software/mbsimAddOn/trunk mbsimAddOn
\end{verbatim}
in directory \texttt{MBSim}. Then one uses the procedures \textsc{automake, configure} and \textsc{install} in the subdirectories of \texttt{MBSim/mbsimAddOn}.

%%------------------------------------------------------------ SUBSECTION ---------------------
\subsection{\MBSim Examples}
The examples are used for testing successful installation. There are two possibilities:
\begin{enumerate}
\item Change to the specific directory \texttt{\$HOME/MBSim/mbsim/examples/*} and type \texttt{make} to create an executable. The simulation starts with the command~\texttt{./main}. The results are visualised with the command~\texttt{openmbv} and plotted with~\texttt{h5plotserie} (cf.~Sec.~\ref{sec:plot}).
\item Use the script \texttt{./runexamples.sh install} in \texttt{\$HOME/MBSim/mbsim/examples} to install reference files. Then \texttt{./runexamples.sh} compiles, runs and tests each example. See \texttt{./runexamples.sh -h} for additional information.
\end{enumerate}

